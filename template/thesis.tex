%
% Adaptation of the Thesis Template bei Philip Döbler shared under CC-BY 3.0
% Ergänzungen von:
% Till Tantau (teilweise entfernt)
% André Calero Valdez
% Tim Schrills


\documentclass[
    paper = a4,
    fontsize = 12pt,
    headinclude = true,
    % start chapters on the right side
    open = right,
    % Use twosided layout? Every chapter starts on the right side, therefore sometimes blank pages are added between chapters.
    twoside = false,
    BCOR = 10mm,
    % add lists and bibliography to table of contents
    toc = listofnumbered,
    toc = bibnumbered,
    % enumerate chapters as x.y instead of x.y.
    numbers = noendperiod
]{scrreprt}

% use some additional package, add your own below if required
\usepackage[utf8]{inputenc}
\usepackage[T1]{fontenc}
% package for code listings
\usepackage{listings}
% package providing colors (e.g. for syntax highlighting in listings)
\usepackage{xcolor}
% package for german language, change to english if required
\usepackage[ngerman]{babel}
\usepackage[german=quotes]{csquotes}
\usepackage{xpatch}
\usepackage{hyphenat}
% package to set the line spacing
\usepackage{setspace}
% package for hyperlinks (e.g., in the table of contents) and set links to same style as text
\usepackage[hidelinks]{hyperref}
% package to create acronyms
\usepackage[acronym]{glossaries}
% package for pseudocode
\usepackage[german, algochapter, linesnumbered]{algorithm2e}
% package to create diagrams
\usepackage{tikz}
\usetikzlibrary{positioning}
\usetikzlibrary{shapes.geometric, arrows}
\tikzset{font={\fontsize{10pt}{12}\selectfont}}

%%% additions by ACV
% Citations using APA style with biblatex
\usepackage[style=apa,backend=biber,backref=true,maxbibnames=20]{biblatex}
% Biblatex allows multiple bib-files
\addbibresource{bibliography/references.bib}
\addbibresource{bibliography/websources.bib}
\addbibresource{bibliography/normen.bib}

% Tables using APA Style
\usepackage{booktabs}

% use microkerning for improved typesetting
\usepackage{microtype}

% include acronyms
% add acronyms below
\newacronym{nic}{NIC}{Network Interface Controller}


% avoid bold caption for algorithms to be consistent with other captions
\SetAlCapSty{}
\SetNlSty{footnotesize}{}{}

% definitions for syntax highlighting in listings environment
% define custom colors for syntax highlighting in code listings
\definecolor{lst_comments}{rgb}{0, 0.75, 0}
\definecolor{lst_linenumbers}{rgb}{0.2, 0.2, 0.2}
\definecolor{lst_keywords}{rgb}{0, 0, 0.75}
\definecolor{lst_strings}{rgb}{0.75, 0, 0}
\definecolor{lst_background}{rgb}{0.97, 0.97, 0.97}

% define style for listings
\lstdefinestyle{msqc_style}{
    backgroundcolor=\color{lst_background},   
    commentstyle=\color{lst_comments},
    keywordstyle=\color{lst_keywords},
    stringstyle=\color{lst_strings},
    % style of the line numbers on the left side
    numberstyle=\tiny\color{lst_linenumbers},
    % use monospace font for code and set size to the size of footnotes
    basicstyle=\footnotesize,
    xleftmargin=2em,
    % break long lines, if this happens it might be better to manually add a line break at an appropriate place
    breaklines = true,
    % place caption below the listing
    captionpos = b,
    % don't drop spaces to fix alignment and always convert tabs to spaces
    keepspaces = true,
    % place line numbers on the left side
    numbers = left,
    % distance between line numbers and listing
    numbersep = 7pt,
    % don't show special characters for spaces
    showspaces = false,
    % don't show special characters for spaces in strings
    showstringspaces = false,
    % don't show special characters for tabs
    showtabs = false,
    % set size of tab to 4 spaces
    tabsize = 4
}
% set style for the document
\lstset{style=msqc_style}

% set line spacing to 1.5
\onehalfspacing

\begin{document}
    % use roman numbering for preamble
    \pagenumbering{Roman}
    % add title page and abstract
    \input{preamble/title_page}
    % needed hear to properly insert blank pages for double page layout
\cleardoublepage

% use two mini pages for abstract
\begin{minipage}[0.95\textheight]{0.95\textwidth}
    \large
    \textbf{Kurzfassung}\par
    \vspace{\baselineskip}
    \small
    Der Abstract einer Abschlussarbeit sollte eine kurze Zusammenfassung enthalten, damit der Leser nach einigen Sätzen einen Eindruck davon bekommt, welches Thema bearbeitet wurde.
Ein Abstract ist dabei kein \textquote{Teaser} sondern eher eine \textquote{Executive Summary}.

Dieses Dokument dient als Vorlage und gleichzeitig als kleine Anleitung, um eine Abschlussarbeit mit \LaTeX{} zu erstellen. Um das Template für die eigene Abschlussarbeit zu verwenden, kann einfach der vorhandene Text gelöscht und eigener Text hinzugefügt werden. Das Dokument enthält keine ausführliche Erklärung für das Arbeiten mit \LaTeX{}, da es hierzu eine Vielzahl von Tutorials im Internet gibt. Stattdessen enthält es einige Tipps und Richtlinien. Der Quellcode ist ausführlich dokumentiert, damit es einfach ist das Template für die eigene Arbeit anzupassen.

    
    \vspace{\baselineskip}
    \normalsize
    \textbf{Schlüsselwörter}\par
    \vspace{0.5\baselineskip}
    \small    
    Schlüsselwort 1\\
Schlüsselwort 2\\
\end{minipage}

%\vspace{1cm}
\newpage

\begin{minipage}[0.95\textheight]{0.95\textwidth}
    \large
    \textbf{Abstract}\par
    \vspace{\baselineskip}
    \small
    a short English description of the thesis


    \vspace{\baselineskip}
    \normalsize
    \textbf{Keywords}\par
    \vspace{0.5\baselineskip}
    \small
    \input{preamble/keywords_eng}
\end{minipage}

    
    % print table of contents
    \tableofcontents
    
    % use arabic numbering for main part
    \cleardoublepage\pagenumbering{arabic}

    % Deutsche Paragrapheinzüge auf 0 und Parskip auf 10 pt
    % Für englische Text auskommentieren
    \setlength{\parindent}{0pt}
    \setlength{\parskip}{10pt}
    
    % create a tex file for each chapter and include it below

    %%%%%%%%%%%%%%%%%%%%%%%%%%%%%%%%
    %%%%%%%%%%%%%%%%%%%%%%%%%%%%%%%% 
    % Inhalt der Arbeit
    \chapter{Einleitung}
Die Einleitung erklärt den Kontext der eigenen Arbeit und führt zur Fragestellung hin, die bearbeitet wurde. Es sollte klar werden, in welchem Bereich die Arbeit verfasst wurde und warum sie relevant ist. Im Gegensatz zum Abstract wird die Arbeit hier nicht zusammengefasst. Am Ende der Einleitung kann der Aufbau der restlichen Arbeit erläutert werden.

\section{Struktur der Arbeit}

Es gibt unterschiedliche Strukturen, wie eine Qualifizierungsarbeit aufgebaut sein kann. Es ist daher sinnvoll, die Struktur der eigenen Arbeit mit der Betreuer:in zu besprechen. 

%
% Generelle Hinweise:
% - Werfen Sie auch einen Blick in die Word-Vorlage, falls dort Hinweise sind, die hier nicht enthalten sind.
%-	(Dieses Dokument ist für einseitigen Druck formatiert; wenn zweiseitig gedruckt werden soll, müssen die Seitenzahlen und Header entsprechend angepasst werden.) 
%-	Auf Abbildungen / Tabellen wird möglichst im Text vor der Abbildung verwiesen. Ist in Latex manchmal schwierig
%-	Abbildungen sollten nach Möglichkeit so groß dargestellt sein, dass auch die Texte gut lesbar sind; es sei denn diese sind völlig bedeutungslos und nur die Struktur oder das Gesamtbild sind von Bedeutung.
%-	Bei farbigen Abbildungen sollte sichergestellt werden, dass diese auch in Schwarz-Weiß gut erkennbar sind.
%-	Tabellen sollten zweckmäßig und übersichtlich sein: Vermeidung unnötiger Linien, Farbgebung nur, wenn sie eine Bedeutung hat oder der Übersichtlichkeit dient.
%-	Zitiert wird typischerweise nach APA. Alternativen sind aber möglich (mit den Betreuer:innen klären). 


    \chapter{Verwandte Arbeiten oder Stand der Forschung}
Der Stand der Forschung oder das Kapitel \enquote{Verwandte Arbeiten} dient dazu, die Forschungsfrage/Hypothesen herzuleiten und den Gegenstand der eigenen Arbeit in ein größeres Forschungsfeld einzubetten. Dabei sollten fremde Arbeiten nicht einfach \enquote{aufgelistet} werden sondern sollten inhaltlich diskutiert (und ggfs.~kritisiert) werden. 
Kann ggfs. auch als Unterkapitel des ersten Kapitels eingegliedert werden.
    \chapter{Weitere Kapitel}
Natürlich enthält die Arbeit noch weitere Kapitel. Welche genau für Deine Arbeit wichtig sind, hängt von der Art der Arbeit ab und sollte mit der Betreuer:in besprochen werden.

\section{Unterkapitel}

\subsection{Unterunterkapitel}

\subsubsection{Unterunterunterkapitel}
    %\include{chapters/40_method}
    %\include{chapters/50_results}
    %\include{chapters/60_discussion}
    \chapter{Zusammenfassung und Ausblick}
Im letzten Kapitel sollte die Arbeit zusammengefasst und ein Fazit gezogen werden. Außerdem sollte beschrieben werden, wie es mit dem Projekt weitergehen kann und welche Punkte vielleicht interessant wären aber im Rahmen der Arbeit nicht bearbeitet werden konnten.

    % diese Zeilen auskommentieren, um das Tutorial aus der Arbeit auszublenden.
    \chapter{Tutorial zu dieser LaTeX-Vorlage}
Dieses Kapitel ist spezifisch für die \LaTeX Vorlage und sollte natürlich in der finalen Abgabe nicht enthalten sein.
Dieser Teil der Arbeit kann bei Bedarf durch einen Kommentar einfach ausgeblendet werden (siehe thesis.tex Zeile 125).






\section{Verwendung dieser Vorlage}
Dieses Template ist für die Verwendung mit pdflatex gedacht. Am einfachsten ist es die Vorlage in Overleaf \cite{overleaf} zu öffnen. Overleaf ist eine Onlineanwendung zum Arbeiten mit \LaTeX{}, was den Vorteil hat, dass nichts lokal installiert werden muss und mit jedem Betriebssystem gearbeitet werden kann, das über einen Browser verfügt. Außerdem ist es möglich mit mehreren Personen gemeinsam an einem Projekt zu arbeiten. Die Vorlage kann auch mit einer lokalen Installation verwendet werden. Die Website von Overleaf bietet zudem einige gute Tutorials zum Arbeiten mit \LaTeX{}: \url{https://www.overleaf.com/learn}.

Fehler und Warnungen, die beim Compilieren erzeugt werden, sollten direkt behoben werden, da es später schwierig sein kann den eigentlichen Auslöser einer Fehlermeldung zu finden. Manchmal sieht das Dokument trotz Fehlermeldung oder Warnung korrekt aus, der Fehler macht sich dann aber später bemerkbar. Die Meldungen sind leider oft nicht sehr aussagekräftig, weshalb es am einfachsten ist direkt nach dem Auftreten eines Fehlers den Teil der Arbeit anzuschauen, der als letztes geändert wurde.

\section{Projektstruktur}

Der Hauptteil einer Thesis besteht üblicherweise aus mehreren Kapiteln, die verschiedene Aspekte der Arbeit beleuchten. 
Es ist ratsam für jedes Kapitel ein eigene Tex-Datei anzulegen, damit der Quellcode übersichtlich bleibt. Durch einen numerischen Präfix (z.B.: \texttt{20\_relatedwork.tex} siehe Abb. \ref{fig:folderstructure}) werden die Quelldateien in der richtigen Reihenfolge in Overleaf angezeigt. Wir verwenden 20 anstatt 2, damit wir nachträglich auch noch 21, 22, etc. einfügen können.

\begin{figure}
    \centering
    \includegraphics[width=0.4\linewidth]{figures/folderstructure.png}
    \caption{Ordnerstruktur eines LaTeX-Projektes}
    \label{fig:folderstructure}
\end{figure}

\subsection{Unterkapitel}
Unterkapitel sollten ein abgeschlossenes Thema behandeln. Einzelne Unterkapitel in einem Kapitel sind zu vermeiden, also z.\,B. in Kapitel 2 das Unterkapitel 2.1, aber kein weiteres Unterkapitel. In diesem Fall ist es besser entweder den Inhalt von 2.1 direkt in Kapitel 2 zu schreiben, oder falls 2 und 2.1 thematisch zu weit voneinander entfernt sind, aus Unterkapitel 2.1 ein eigenes Kapitel 3 zu machen. Dieses Unterkapitel ist ein negativ Beispiel dafür. 

\section{Grafiken}
In der Informatik sind die häufigsten Grafiken entweder Diagramme oder Plots. Beide Arten von Grafiken lassen sich gut als Vektorgrafiken erstellen und einbinden. Der Vorteil von Vektor- gegenüber Pixelgrafiken ist, dass beliebig weit in eine Grafik hereingezoomt werden kann, ohne dass sie unscharf wird. Zudem benötigen Vektorgrafiken meistens weniger Speicherplatz.

\subsection{Vektor- vs Pixelgrafiken}
Für die meisten Abbildungen sollte man Vektorgrafiken verwenden, diese können verlustfrei skaliert werden und bieten somit die meisten Freiheiten.
Wenn man Grafiken in R erstellt, können die Plots als pdf oder svg-Dateien abgespeichert werden und liegen dann ebenfalls als Vektorgrafik vor. Ein weiteres beliebtes Programm zum Erstellen von Vektorgrafiken ist Inkscape \cite{inkscape}. Zudem bieten viele Programme die Möglichkeit eine Grafik z.\,B. als PDF zu exportieren, was in \LaTeX{} als Vektorgrafik eingebunden werden kann. Adobe Indesign wird auch häufig zum Erstellen von Vektorgrafiken verwendet.

\paragraph{Profi-Tipp TikZ.}
Grafiken können direkt in \LaTeX{} mit dem TikZ Paket \cite{tikz} erstellt werden. Die Verwendung ist etwas gewöhnungsbedürftig, da Grafiken mit Code beschrieben werden, bietet aber viele Freiheiten. Außerdem werden die so erstellten Grafiken direkt in \LaTeX{} gerendert und verwenden die selbe Schriftart wie im Text und eine konsistente Schriftgröße im gesamten Dokument. 

Pixelgrafiken lassen sich nicht immer vermeiden, z.\,B. wenn eine Foto in die Arbeit eingebunden werden soll. In diesem Fall sollte darauf geachtet werden, dass die Grafik über eine ausreichende Auflösung verfügt. Eine Auflösung von 300~dpi ist ein guter Richtwert, um beim Drucken ein gutes Ergebnis zu erhalten.

Abbildungen \ref{fig:ieee_float_format_vector} und \ref{fig:ieee_float_format_pixel} zeigen beide den Aufbau des IEEE Floating Point Formats. Abbidlung \ref{fig:ieee_float_format_pixel} ist eine Pixelgrafik, während Abbildung \ref{fig:ieee_float_format_vector} mit TikZ erstellt wurde. Der Unterschied wird beim hereinzoomen deutlich.

\begin{figure}[ht]
\centering
% Beispiel für eine mit TIKZ erstellte Grafik

\begin{tikzpicture}
[node distance=0cm]
\tikzset{rectangle_black/.style = {rectangle, minimum width=1cm, minimum height=0.75cm, text centered, draw=black, fill=white}};
\tikzset{rectangle_white/.style = {rectangle, minimum width=1cm, minimum height=0.75cm, text centered, draw=none, fill=none}};
\node (sign) [rectangle_black] {S (sign)};
\node (exponent) [rectangle_black, right = of sign] {E (biased exponent)};
\node (fraction) [rectangle_black, right = of exponent] {T (trailing significand field)};
\node (sign_text) [rectangle_white, above = of sign] {1};
\node (exponent_text) [rectangle_white, above = of exponent] {$w$};
\node (fraction_text) [rectangle_white, above = of fraction] {$t=p-1$};
\node (bits) [rectangle_white, above = of fraction, right = of fraction_text] {bits};

\end{tikzpicture}

\caption{Aufbau des IEEE Floating Point Formats als Vektorgrafik mit TikZ erzeugt.}
\label{fig:ieee_float_format_vector}
\end{figure}

\begin{figure}[ht]
\centering
\includegraphics[width=0.65\textwidth]{figures/ieee_float_format.jpg}
\caption{Aufbau des IEEE Floating Point Formats als Pixelgrafik.}
\label{fig:ieee_float_format_pixel}
\end{figure}

\section{Tabellen}
Tabellen können in \LaTeX{} direkt erstellt werden. Tabelle \ref{tab:ieee_formats} zeigt ein Beispiel dafür. Einfache Tabellen lassen sich schnell erstellen, bei komplizierteren Tabellen ist es manchmal einfacher zusätzliche Pakete zu verwenden. Mit dem Paket \texttt{multirow} können z.\,B. einfacher Tabellen erstellt werden, bei denen einzelne Zeilen oder Spalten zusammengefasst sind. 

Unter \url{https://www.tablesgenerator.com/} findet man ein hilfreiches online Werkzeug zum erstellen von Tabellen. Wichtig ist es hier den \textquote{Default table style} zum \textquote{Booktabs table style umzustellen}. 

\begin{table}[ht]
\centering
\begin{tabular}{lrrrr} 
 \toprule
 Parameter & binary16 & binary32 & binary64 & binary128 \\
 \midrule
  $k$, storage width in bits           & 16 &  32 &   64 &   128 \\ 
  $w$, exponent field width in bits    &  5 &   8 &   11 &    15 \\
  $t$, significand field width in bits & 10 &  23 &   52 &   112 \\
  emax, maximum exponent $e$           & 15 & 127 & 1023 & 16383 \\
  bias, $E-e$                          & 15 & 127 & 1023 & 16383 \\
 \bottomrule
\end{tabular}
\caption{IEEE 754-2019 Floating Point Formate als Beispiel für das Einbinden einer Tabelle.}
\label{tab:ieee_formats}
\end{table}

\section{Quellcode}
Um Quellcode in die Arbeit einzubinden, können in \LaTeX{} Listings verwendet werden. Es gibt für populäre Sprachen vorgefertigte Umgebungen, welche die Syntax farblich hervorheben. Quellcode sollte eingebunden werden, wenn eine konkrete Implementierung in einer Sprache erläutert wird. Für die Erklärung eines Algorithmus ist es oft übersichtlicher ein Schaubild oder Pseudocode zu verwenden. Es sollten nur kurze Codeabschnitte eingebunden werden, die für den Leser einfach nachvollziehbar sind und nur den für die Erklärung relevanten Code enthalten. Längere Codeabschnitte können im Anhang stehen. Der komplette Code, der für die Arbeit geschrieben wurde, sollte in einem Repository (Gitlab) % oder Github)
abgelegt werden.

Listing~\ref{lst:example_listing} zeigt ein Beispiel für ein Codelisting in der Programmiersprache C. Algorithmus \ref{alg:west} zeigt einen Routing Algorithmus als Pseudocode. Der Code wurde mit dem Paket algorithm2e \cite{algorithm2e} erstellt.

\begin{lstlisting}[language=C, caption=Beispiel für ein Codelisting in der Sprache C., label=lst:example_listing]
#include <stdio.h>
// comments are highlighted in green
void main() {
    // keywords of the language are highlighted in blue
    for (int i = 0; i <= 42; ++i) {
        printf("%d\n", i);
    }
    // strings are highlighted in red
    printf("Hello World!");
}
\end{lstlisting}

\begin{algorithm}[ht]

    \uIf{destination in west direction}{
        go West;
    }
    \uElseIf{destination in same column}{
        \eIf{destination in north direction}{
            go North;
        }{
            go South;
        }
    }
    \uElseIf{destination in north east direction}{
        go North or go East
    }
    \uElseIf{destination in south east direction}{
        go South or go East
    }
    \uElseIf{destination in same row and in east direction}{
        go East;
    }
    \uElseIf{at destination}{
        done;
}
 \caption{West First-Routing Algorithm.}
 \label{alg:west}
\end{algorithm}

\section{Literatur}
Ein Literaturverzeichnis sollte mit dem apa Paket für BibLatex erstellt werden. Dazu wird für jede Quelle ein Eintrag in der Datei \texttt{references.bib} angelegt. An der passenden Stelle im Text können diese Einträge mit dem \texttt{\textbackslash cite\{\}} Befehl zitiert werden. Für jede Quelle die zitiert wird, legt \LaTeX{}  im Literaturverzeichnis einen Eintrag an.

Beschreibungen der Quellen im Bibtex-Format müssen meistens nicht selbst erstellt werden, sondern können direkt bei vielen Verlagen und Bibliotheken direkt generiert werden. Google bietet mit dem \textquote{Scholar-Button} ein Chromium Plugin, mit dem schnell bibtex-Einträge generiert werden können. Bei Google-Scholar generierten Bibtex-Einträgen muss auch eine manuelle Endkontrolle stattfinden, um zu prüfen, ob die Daten in Google korrekt gespeichert waren (z.B.~fehlende Autoren, falsche Jahreszahl, etc.).

Es gibt verschiedene Arten wie man mit biblatex zitiert.\\
\noindent
\texttt{\textbackslash autocite\{valdez2015reducing\}} führt zu \autocite{valdez2015reducing}.\\
\texttt{\textbackslash parencite\{valdez2015reducing\}} führt zu \parencite{valdez2015reducing}.\\
\texttt{\textbackslash textcite\{valdez2015reducing\}} führt zu \textcite{valdez2015reducing}.\\
\texttt{\textbackslash cite\{valdez2015reducing\}} führt zu \cite{valdez2015reducing}.\\
\texttt{\textbackslash autocite[siehe auch][S. 13]\{valdez2015reducing\}} führt zu \autocite[siehe auch][S.~13]{valdez2015reducing}.\\

Hier\footnote{\url{https://tug.ctan.org/info/biblatex-cheatsheet/biblatex-cheatsheet.pdf}} gibt es ein hilfreiches Cheat-Sheet.

Je nach zitierter Dokumentsorte, sieht die Referenz im Literaturverzeichnis anders aus.
\begin{itemize}
    \item Beispiel für einen Konferenzbeitrag \autocite{Nielsen1990}
    \item Beispiel für einen Journal-Artikel \autocite{hollan2000}
    \item Beispiel für ein Buch~\autocite{zobel2014writing}.
    \item Beispiel für eine Norm~\autocite{ISO9241}.
    \item Beispiel für einen Weblink\footnote{{\url{http://imis.uni-luebeck.de}} \autocite{webimis}}
\end{itemize}

\textbf{Hinweis}: Keine dieser Referenzen müssen Sie in Ihrer Arbeit zitieren!

\section{Abkürzungen}
Für jede verwendete Abkürzung kann ein Eintrag in der Datei \texttt{acronyms.tex} angelegt werden. Wenn diese Abkürzung im Text zum ersten Mal auftaucht, sollte der Begriff ausgeschrieben werden mit der Abkürzung in Klammern dahinter. Bei weiteren Vorkommen im Text kann dann die eigentliche Abkürzung verwendet werden. In \LaTeX{} gibt es dafür spezielle Befehle. Beispiel für ausgeschriebene Abkürzung (siehe Quelltext des Dokuments für die entsprechenden Befehle).

Erste Verwendung mit Erläuterung: \acrfull{nic}.\\ 
Beispiel für das Verwenden der Abkürzung: \acrshort{nic}. 

Die verwendeten Abkürzungen werden automatisch im Abkürzungsverzeichnis aufgelistet.



    

\section{LaTeX Eigenarten}
Hier noch ein paar latexspezifische Dinge beim Schreiben.

\paragraph{Non-Breaking Space.}
Mit dem Tildezeichen $\sim{}$ bekommt man ein sog. non-breaking Space~(nbsp). Das ist hilfreich, wenn man verhindern möchte, dass z.B. die Zeile mit einer Referenz beginnt (Beispiel: Listing~\ref{lst:example_listing} -- Hier wird Listing niemals von der Zahl getrennt werden).
Das nbsp wird auch verwendet, um den Abstand nach einem Punkt in einer Abkürzung (1 Einheit) nicht auf die Länge des Abstandes nach dem Satzende zu setzen (1,5 Einheiten). Das fällt nicht immer jedem auf, wenn man es aber einmal sieht, kann man es schwer wieder abschalten.\\
Beispiel: Dies ist eine Abk.~in einem Satz. (korrekt) Die Leertaste wird hier immer gleich lang gehalten.\\
Beispiel: Dies ist eine Abk. in einem Satz. (falsch) Die Leertaste kann (!) hier vom Setzer verlängert werden, um die Satztrennung deutlicher zu machen.\\

\paragraph{Binde- und Gedankenstriche.}
 Es gibt drei Stricharten in Latex. Diese haben unterschiedliche Bedeutungen.
 Der Bindestrich~-~(engl. hyphen) ist ein einfaches Minus und wird verwendet, um Trennung von Silben anzuzeigen oder Mehrsprachige Komposita zu ermöglichen (z.B. Dashboard-Anzeige). Siehe auch \url{https://www.duden.de/sprachwissen/rechtschreibregeln/bindestrich}.
 
 Der deutsche Gedankenstrich (engl. en-dash) -- der selten (!) bei Einschüben verwendet wird -- sind zwei Minuszeichen und wird mit Leerzeichen (oder nbsp) abgesetzt. Siehe auch \url{https://www.duden.de/sprachwissen/rechtschreibregeln/gedankenstrich}.
 Im Englischen wird der en-dash u.a. dazu benutzt, Zahlenbereiche zu beschreiben: z.B.~pages 4--9.

 Der englische Gedankenstrich --- (engl. em-dash) sind drei Minuszeichen und wird im Englischen für Einschübe benutzt und wird \textbf{nicht} mit Leertasten abgesetzt. Beispiel:
 People---at least most of them---are suprised at how the em-dash is used properly. Im Deutschen kommt dieser Gedankenstrich nicht vor. 
 Verwendung im Englischen siehe auch: \url{https://www.merriam-webster.com/words-at-play/em-dash-en-dash-how-to-use}

\paragraph{Anführungszeichen.}
Wichtig: Anführungszeichen im Deutschen sind anders als im Englischen und Französischen. 

Deutsche Anführungszeichen -- nach innen gewölbte doppelte Tief- und Hochkommata~-- werden mit Anführungszeichen und Backtick ("{}\`{}) geöffnet und mit Anführungszeichen und Apostroph ("{}'{}) geschlossen.
Beispiel: 
"`Latex sollte bei korrekt eingestellter Dokumentsprache aber die korrekten Anführungszeichen wählen."' sagte er leichtsinnig und irrte sich.

\url{https://www.duden.de/sprachwissen/rechtschreibregeln/anfuehrungszeichen}

Englische Anführungszeichen -- nach außen gewölbte doppelte Hochkommata -- werden mit doppeltem Backtick geöffnet (\`{}\`{}) und mit doppeltem Apostroph ('{}'{}) geschlossen. Beispiel:
``This is a correct direct citation in British English''. Der Punkt steht im englischen Zitat ausserhalb der Anführungszeichen (nicht im American English).

Mit dem Command \textbackslash flqq erhält man das einleitende französische Anführungszeichen~(\flqq). Mit \textbackslash frqq{} bekommt man das Gegenstück (\frqq).

    % add bibliography
    %\printbibheading
    % Die Kategorien können hier angepasst werden.
    \printbibliography[nottype=online,notkeyword=norm]
    \printbibliography[type=online,heading=subbibliography,
title={Online-Quellen}]
    \printbibliography[keyword=norm,heading=subbibliography,
title={Normen}]
    
    \begin{appendix}
        % create a tex file for each appendix and include it below
        \chapter{Anhang}
Der Anhang kann Teile der Arbeit enthalten, die im Hauptteil zu weit führen würden, aber trotzdem für manche Leser interessant sein könnten. Das können z.\,B. die Ergebnisse weiterer Messungen sein, die im Hauptteil nicht betrachtet werden aber trotzdem durchgeführt wurden. Es ist ebenfalls möglich längere Codeabschnitte anzuhängen. Jedoch sollte der Anhang kein Ersatz für ein Repository sein und nicht einfach den gesamten Code enthalten.
        
        % add lists of figures, tables and listings to the appendix
        \listoffigures
        \listoftables
        \lstlistoflistings
        
        % add acronyms
        \printglossary[type=\acronymtype, title=Abkürzungsverzeichnis, toctitle=Abkürzungsverzeichnis, numberedsection] 
        

    \end{appendix}
    \newpage
    \Large
Erklärung
\newline
\vspace{3\baselineskip}
\normalsize
\noindent

Ich versichere, die vorliegende Arbeit selbstständig verfasst und nur die angegebenen Quellen benutzt zu haben.


\vspace{5\baselineskip}


Unterschrift 

Lübeck, Tagesdatum


    %%%%%%%%%%%%%%%%%%%%%%%%%%%%%%%%
    %%%%%%%%%%%%%%%%%%%%%%%%%%%%%%%% 
    
\end{document}
