\chapter{Projektorganisation}
Am ersten Termin des Projektes wurde die Projektaufbauorganisation erstellt und die Verantwortlichkeiten und Befugnisse der Aufgabenträger definiert. Die Organisation folgt dabei der Reinen Projektorganisation, da die Teammitglieder Studenten sind. Eine Linien- oder Matrixorganisation durch die fehlende Unternehmensorganisation nicht gegeben. Durch die Organisation sind die Vorteile der reinen Projektorganisation im Team spürbar gewesen. Die Motivation war hoch, es konnte schnell auf Änderungswünsche reagiert werden und es kam zu keinen nennenswerten Konflikten. In Unternehmen müsste das Projektteam aufgelöst werden. Dieser Nachteil entfällt. 

Die Einteilung der Ämter innerhalb der Studiengruppe ist in \autoref{chap:Studiengruppe} beschrieben. Die Projektablauforganisation folgt dem V-Modell. Die einzelnen Projektabschnitte wurden dabei, wie bereits erwähnt, vom Auftraggeber vorgegeben und die Feinplanung mithilfe eines Projektstrukturplans durchgeführt. In diesem sind die einzelnen Arbeitspakete genannt und zeitlich abgeschätzt worden. Auf dieser Grundlage wurde die Zeitplanung erarbeitet. Anschließend folgte die Beschreibung der Arbeitspakete mit Inputs und Outputs, sowie einer verbesserten zeitlichen Abschätzung. Im folgenden sind die Ergebnisse dieser Planungen beschrieben.
\section{Projektstrukturplan}
\subsection{Zeitplanung}
\subsection{Meilensteintrendanalyse}
\section{Anforderungsliste}