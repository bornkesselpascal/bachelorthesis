\chapter{Einführung}

In den Studiengängen Flug- und Fahrzeuginformatik (FFI) und Luftfahrttechnik (LT) gibt es im 6ten Studiensemester ein verpflichtendes Projekt. Dieses wird von Projektteams, die aus FFI und LT Studenten besteht, absolviert. Die Inhalte des Projektes werden generisch in den Modulhandbüchern beider Studiengänge beschrieben. Im Modulhandbuch des Studiengangs FFI wird der Inhalt im Entwicklungszyklus des V-Modells beschrieben. Die genannten Schritte sind:
\begin{enumerate}
	\setstretch{1}
	\item Analyse
	\item Recherche zur Lösungsvorbereitung
	\item Beschreibung der Lösung
	\item Auswahl von Methoden und Tools
	\item Implementierung
	\item Verifikation
	\item Erstellung eines Abschlussberichts
	\item Projektbegleitendes Projekt- und Konfigurationsmanagement
\end{enumerate}
\setstretch{1.63}
Im Modulhandbuch des Studiengang LT wird der Inhalt konkreter beschrieben. Das Projekt ist auf dem Gebiet der Luftfahrttechnik und wird als Teamarbeit gelöst. Die wechselnden Randbedingungen der Flugmissionen werden bereits erwähnt. In der verpflichtenden Literatur ist das ausgegebene Lastenheft aufgeführt, welches den zu erbringenden Leistungsumfang definiert.
Das ausgegebene Projekthandbuch konkretisiert das Projektziel mit den Worten: Das Projektziel ist die Erstellung einer senkrecht startenden und landenden Flugdrohne aus verfügbaren und selbst gefertigten Bauelementen sowie der Entwicklung, Fertigung und Integration einer Vorrichtung zur Aufnahme, Beförderung und Absetzen von Lasten. Die Funktion des Systems ist im Rahmen einer vorgegebenen Flugmission nachzuweisen.

\section{Hintergrund des Themas}
Das Ziel der Vorlesung wird ebenfalls in den Modulhandbüchern erläutert. Beide Modulhandbücher legen Wert auf eine selbstständige Teamarbeit um eine Gesamtlösung zu erarbeiten. Es geht um das vollumfängliche Arbeiten in einer Projektumgebung, inklusive des Projektmanagements, der Organisation und der Dokumentation. Die angestrebten Lernergebnisse des Studiengangs LT erwähnen die ingenieurwissenschaftlichen Methoden im besonderen.

Das Projekthandbuch ist im Stile eines Lastenheftes geschrieben. In Kapitel 2 werden allgemeine Definitionen und Rahmenbedingungen beschrieben. Das Projekt wird innerhalb eines Produktentstehungsprozesses (PEP) bearbeitet. Der PEP richtet sich nach dem NASA Systems Engineering Handbook (SP-2016-610S Rev. 2). In diesem Enthalten sind mehrere vorzubereitende Technische Reviews (PDR, CDR, FRR). Im Überblick bleibt zu sagen das in diesem Projekt der Entwicklungszyklus eines Fluggerätes im kleinen Simuliert wird. Das Ziel ist es den Studenten einen Einblick in die Entwicklungsprozesse von luftfahrttechnischen Geräten zu geben. Dazu gehört die abschließende Ausgabe einer THI-Registriernummer und die Erteilung der Permit-to-fly (PTF).

\section{Zusammenfassung} %Problemstellung
Die Aufgabenstellung des Flugprojektes im Sommersemester 2023 (SS23) beinhaltet drei Flugaufgaben (Challenges). Für Challenge \#1 ist eine flugfähige Drohne zu erstellen, die eine definierte Mission autonom abfliegt. Für Challenge \#2 ist eine angebrachte Last mithilfe einer konstruierten Lastaufhängung autonom an einen definierten Ort zu bringen und abzusetzen. Für Challenge \#3 ist die Last maximal oft zu transportieren. Die Einzelnen Challenges werden in späteren Kapiteln weiter beschrieben. Das Team Blau hat hierfür eine Drohne und Lastaufhängung entwickelt. Die Drohne wurde als Quadrocopter ausgelegt und besitzt vier Arme aus Forged Carbon. Die Lastaufhängung ist mithilfe eines Riegels mit Formschluss realisiert, der über einen Servomotor angesteuert wird.